\documentclass{beamer}

\usepackage{amsmath}
\usepackage{textcomp}
\usepackage{listings}
\usepackage{lmodern}

\lstset{
    language=[latex]tex,
    breaklines=true}

%%%%%%%%%%%%%%%%%%% BEGIN : IMPORTED CODE
\lstset{literate={→}{$\rightarrow$}1}
\DeclareUnicodeCharacter{00A0}{ }
\DeclareUnicodeCharacter{2192}{\dash}

\newsavebox{\haskellTypes}
\newsavebox{\pZeroTypes}
\newsavebox{\haskellTypesInst}
\newsavebox{\pZeroTypesInst}

\newsavebox{\anatomyOfCaseBox}
%%%%%%%%%%%%%%%%%%% END : IMPORTED CODE

% \usepackage[utf8]{inputenc}
\usepackage[T1]{fontenc}

\usetheme{Madrid}

\title[Committee Meeting 1]{First Committee Meeting}
\subtitle{Progress Report}
\author{Jason Balaci}
\institute{McMaster University}
\date{Oct. $21^{st}$, 2021}


\AtBeginSection[]
{
  \begin{frame}
    \frametitle{Table of Contents}
    \tableofcontents[currentsection]
  \end{frame}
}

\begin{document}


\frame{\titlepage}

\begin{frame}
\frametitle{Table of Contents}
\tableofcontents
\end{frame}

\section{Introduction}
\begin{frame}
    \frametitle{Who am I?}
    \begin{columns}[T,onlytextwidth]
        \begin{column}{.5\textwidth}
            \begin{minipage}{\textwidth}
                \begin{itemize}
                    \item<1-> Jason Balaci
                    \item<2-> Graduate of \emph{McMaster University}, holding...
                        \begin{itemize}
                            \item<2-> Honours Actuarial and Financial Mathematics (B.Sc.)
                            \item<2-> Minor in Computer Science
                        \end{itemize}
                    \item<3-> Currently pursuing a Master's of Computer Science (M.Sc) at \emph{McMaster University}, under the supervision of \textbf{Dr. Carette}
                \end{itemize}
            \end{minipage}
        \end{column}
        \begin{column}{.45\textwidth}
            \includegraphics[width=.8\textwidth]{me.jpeg}
        \end{column}
    \end{columns}
\end{frame}

\begin{frame}
    \frametitle{Overview of Progression towards Master's of Computer Science (M.Sc.)}
\end{frame}

%% TODO: What am I pursuing
%% TODO: What progress have I made towards my degree regulations
\section{Project}
\subsection{Drasil}
%% TODO: What is Drasil?
%% TODO: What I will be contributing
\subsection{Goal \#1: Typed Expression Language}
\subsection{Goal \#2: Model Discrimination -- ``ModelKinds''}
\section{References}

\begin{frame}
    \frametitle{}
    \framesubtitle{}
\end{frame}


%%%%%%%%%%%%%%%%%%% BEGIN : IMPORTED CODE
\begin{frame}
\frametitle{What are Disjoint Union Types?}

\begin{itemize}
 \item<1-> \textbf{Disjoint union types} (DUTs) are types where values can take the form of one of many type constructions, with a \textbf{unique identifier tag}.
 \item<2-> Each DUT has a set of possible types it can take on, called ``variants'', or ``kinds''.
 \item<3-> These ``variants'' may be records, or the unit type (``()'').
 \item<4-> An instance of a DUT may take on the form of \textbf{only one} of it's variants. 
 \item<5-> They are often found in functional programming languages, where they are usually known as \textbf{Algebraic Data Types} (ADTs).
\end{itemize}

\end{frame}

\begin{lrbox}{\haskellTypes}
\begin{lstlisting}[language=Haskell, basicstyle=\footnotesize]
data List a = Nil | Cons a (List a)
data RGB = Red | Green | Blue
\end{lstlisting}
\end{lrbox}

\begin{lrbox}{\pZeroTypes}
\begin{lstlisting}[language=Pascal, basicstyle=\footnotesize]
type List = Nil | Cons(value: integer, tail: List)
type RGB = Red | Green | Blue
\end{lstlisting}
\end{lrbox}


\begin{lrbox}{\haskellTypesInst}
\begin{lstlisting}[language=Haskell, basicstyle=\footnotesize]
a = Cons 1 (Cons 2 (Cons 3 Nil))
b = Red
\end{lstlisting}
\end{lrbox}

\begin{lrbox}{\pZeroTypesInst}
\begin{lstlisting}[language=Pascal, basicstyle=\footnotesize]
a := Cons(1, Cons(2, Cons(3, Nil())))
b := Red()
\end{lstlisting}
\end{lrbox}

\begin{frame}
\frametitle{What do they look like?}
\begin{itemize}
 \item<1-> Declarations in Haskell,\\
    \usebox{\haskellTypes} \\
    \ \\
 \item<2-> Declarations in our implementation,\\
    \usebox{\pZeroTypes} \\
    \ \\
 \item<3-> Instantiation in Haskell,\\
    \usebox{\haskellTypesInst} \\
    \ \\
 \item<4-> Instantiation in our implementation,\\
    \usebox{\pZeroTypesInst} \\
\end{itemize}

\end{frame}

\begin{frame}
 \frametitle{How do we use DUTs?}
 \framesubtitle{The anatomy of the \texttt{case} statement.}
    \begin{columns}[T,onlytextwidth]
        \begin{column}{.5\textwidth}
            \begin{minipage}{\textwidth}
                \begin{itemize}
                    \item<1-> \texttt{case}s are the only way to access data inside of DUTs.
                    \item<2-> Check if DUTs were \textit{initialized} using an optional \texttt{nil} case at the start.
                    \item<3-> \texttt{case} on any of the variants.
                    \item<4-> Within the statement suite of each variant \texttt{case}, the variable in question is assumed to be an instance of the variant's record.
                    \item<5-> \texttt{default} case allows you to perform either a statement suite or a no-op on all non-covered cases. 
                \end{itemize}
            \end{minipage}
        \end{column}
        \begin{column}{.45\textwidth}
            \begin{onlyenv}
                \begin{minipage}{\textwidth}
                    \setlength{\leftmargini}{0cm}
                    % \usebox{\anatomyOfCaseBox}
                    \begin{itemize}[label={},leftmargin=*]
                     \item[]<1-> \texttt{case <variable> of \{}
                     \item[]<2-> \hphantom{~~~~}\texttt{[nil: <stmtSuite>]}
                     \item[]<3-> \hphantom{~~~~}\texttt{Kind A: <stmtSuite>}
                     \item[]<4-> \hphantom{~~~~}\texttt{Kind B: <stmtSuite>}
                     \item[]<4-> \hphantom{~~~~}\texttt{...}
                     \item[]<5-> \hphantom{~~~~}\texttt{[default: <stmtSuite>]}
                     \item[]<6-> \hphantom{~~~~}\texttt{... or ...}
                     \item[]<7-> \hphantom{~~~~}\texttt{[default nothing]}
                     \item[]<7-> \texttt{\}}
                    \end{itemize}

                    \begin{alertblock}{\footnotesize Cover your \texttt{case}s!}<8->
                        {\footnotesize If you create a non-exhaustive \texttt{case} statement, the compiler will warn you.}
                    \end{alertblock}
                \end{minipage}
            \end{onlyenv}
        \end{column}
    \end{columns}
\end{frame}

%%%%%%%%%%%%%%%%%%% END : IMPORTED CODE

\begin{frame}
\frametitle{References}
\begin{itemize}
 \item Carette, Jacques, Oleg Kiselyov, and Chung-chieh Shan. ``Finally tagless, partially evaluated: Tagless staged interpreters for simpler typed languages.'' \textit{Journal of Functional Programming} 19.5 (2009): 509.
\end{itemize}

\end{frame}


\end{document}
