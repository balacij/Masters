\documentclass{beamer}

\usepackage{amsmath}
\usepackage{textcomp}
\usepackage{listings}
\usepackage{lmodern}
\usepackage{hyperref}
\usepackage[T1]{fontenc}


\lstset{
    language=[latex]tex,
    breaklines=true}

\usetheme{Madrid}

\setbeamertemplate{caption}{\raggedright\insertcaption\par}

\title[Committee Meeting 1]{First Committee Meeting}
\subtitle{Progress Report}
\author{Jason Balaci}

\institute{McMaster University}
\date{Oct. $21^{st}$, 2021}

\AtBeginSection[]
{
  \begin{frame}
    \frametitle{Table of Contents}
    \tableofcontents[currentsection]
  \end{frame}
}

\begin{document}

%%%%%%%%%%%%%%%%%%%%%%%%%%%%%%%%%%%%%%%%%%%%%%%%%%%%%%%%%%%%%%%%%%%%%%%%%%%%%%%
%% TITLE PAGE
%%%%%%%%%%%%%%%%%%%%%%%%%%%%%%%%%%%%%%%%%%%%%%%%%%%%%%%%%%%%%%%%%%%%%%%%%%%%%%%
\frame{\titlepage}

%%%%%%%%%%%%%%%%%%%%%%%%%%%%%%%%%%%%%%%%%%%%%%%%%%%%%%%%%%%%%%%%%%%%%%%%%%%%%%%
%% TABLE OF CONTENTS
%%%%%%%%%%%%%%%%%%%%%%%%%%%%%%%%%%%%%%%%%%%%%%%%%%%%%%%%%%%%%%%%%%%%%%%%%%%%%%%

\begin{frame}
\frametitle{Table of Contents}
\tableofcontents
\end{frame}

%%%%%%%%%%%%%%%%%%%%%%%%%%%%%%%%%%%%%%%%%%%%%%%%%%%%%%%%%%%%%%%%%%%%%%%%%%%%%%%
%% INTRODUCTION
%%%%%%%%%%%%%%%%%%%%%%%%%%%%%%%%%%%%%%%%%%%%%%%%%%%%%%%%%%%%%%%%%%%%%%%%%%%%%%%
\section{Introduction}

\begin{frame}
    \frametitle{Who am I?}
    \begin{columns}[T,onlytextwidth]
        \begin{column}{.5\textwidth}
            \begin{minipage}{\textwidth}
                \begin{itemize}
                    \item<2-> I am \textbf{Jason Balaci}
                    \item<3-> Graduate of \emph{McMaster University}, holding...
                        \begin{itemize}
                            \item<4-> Hons. Actuarial and Financial Mathematics (B.Sc.)
                            \item<5-> Minor in Computer Science
                        \end{itemize}
                    \item<6-> Currently pursuing a thesis-based Master's of Computer Science (M.Sc) at \emph{McMaster University}, under the supervision of \textbf{Dr. Jacques Carette}.
                \end{itemize}
            \end{minipage}
        \end{column}
        \begin{column}{.45\textwidth}<2->
            \begin{figure}
                \includegraphics[width=.8\textwidth]{assets/me.jpeg}
                \caption{Me, Camping in Killarney Prov. Park, Fall 2019}
            \end{figure}
        \end{column}
    \end{columns}
\end{frame}

\begin{frame}
    \frametitle{Overview of Progression Towards C.S. M.Sc.}
    \framesubtitle{Course-related progression}
    \begin{itemize}
        \item<1-> I'm required to complete\footnotemark[1]\footnotemark[2]:
            \begin{itemize}
                \item<2-> One (1) ``Software'' course
                \item<3-> Either of:
                    \begin{itemize}
                        \item<4-> Two (2) ``Theory'' courses, and one (1)``Systems'' course
                        \item<4-> One (1) ``Theory'' course, and two (2) ``Systems'' courses
                    \end{itemize}
            \end{itemize}
        \item<5-> I've completed:
            \begin{itemize}
                \item<6-> CAS 701 ``Logic \& Discrete Mathematics'' - Theory course, Fall 2020
                \item<7-> CAS 761 ``Generative Programming'' - Software course, Fall 2020
                \item<8-> CAS 763 ``Certified Programming with Dependent Types'' - Theory \& Software course, Winter 2021
                \item<9-> COMPSCI 6TB3 ``Syntax-Based Tools and Compilers'' - Systems course, Winter 2021
            \end{itemize}
        \item<10-> Together, the courses completed satisfies the ``Courses Requirement'' as mentioned in the academic calendar\footnotemark[1] and the ``Regulations for the Computer Science M.Sc. Program'' document\footnotemark[2].
    \end{itemize}

    \footnotetext[1]{\tiny\url{https://academiccalendars.romcmaster.ca/preview_program.php?catoid=45&poid=23470&returnto=9166}}
    \footnotetext[2]{\tiny\url{http://www.cas.mcmaster.ca/cas/0files/reg_master_cs_2019a.pdf}}
\end{frame}

\begin{frame}
    \frametitle{Overview of Progression Towards C.S. M.Sc.}
    \framesubtitle{Thesis/research-related Progression}
    \begin{itemize}
        \item<1-> Conducted ``full-time'' research for at least 1 full semester (Spring/Summer 2021), and ``part-time'' research during courses.
        \item<2-> Continuing to research ``full-time''.
        \item<3-> Attended a thesis defence to learn about what to expect from a thesis defence (and learn about their research).
        \item<4-> Supervisory committee is formed, and we are currently having our first supervisory committee meeting.
            \begin{itemize}
                \item \emph{Supervisor}: Dr. Jacques Carette
                \item Dr. Spencer Smith
                \item Dr. Wolfram Kahl
            \end{itemize}
    \end{itemize}
\end{frame}

%%%%%%%%%%%%%%%%%%%%%%%%%%%%%%%%%%%%%%%%%%%%%%%%%%%%%%%%%%%%%%%%%%%%%%%%%%%%%%%
%% PROJECT
%%%%%%%%%%%%%%%%%%%%%%%%%%%%%%%%%%%%%%%%%%%%%%%%%%%%%%%%%%%%%%%%%%%%%%%%%%%%%%%

\section{Project}
\subsection{Drasil}

\begin{frame}
    \frametitle{Preface}
    \framesubtitle{What is Drasil?}
    \begin{columns}[T,onlytextwidth]
        \begin{column}{.5\textwidth}
            Drasil...
            \newline \newline
            \begin{minipage}{\textwidth}
                \begin{itemize}
                    \item<2-> is managed by Dr. Carette \& Dr. Smith.
                    \item<3-> originates from the work of Dan Szymczak.
                        \begin{itemize}
                            \item<4-> Originally focused on scientific software (\emph{Literate Scientific Software}).
                            \item<5-> Focus expanded...
                        \end{itemize}
                    \item<6-> tries to ``Generate All The Things''...
                    \begin{itemize}
                        \item<7-> with a focus on research software.
                    \end{itemize}
                    \item<8-> has a website\footnotemark[1]!
                \end{itemize}
            \end{minipage}
        \end{column}
        \begin{column}{.45\textwidth}
            \begin{figure}
                \includegraphics[width=.8\textwidth]{assets/drasil-logo.png}
                \caption{Drasil's Logo \tiny\cite{Drasil2021}\cite{YggdrasilWiki2021}}
            \end{figure}
        \end{column}
    \end{columns}

    \footnotetext[1]{\tiny \url{https://jacquescarette.github.io/Drasil/}}
\end{frame}

\begin{frame}
    \frametitle{Drasil}
    \framesubtitle{``Generate All The Things!''}
% TODO: ``Generate All The Things!'' is a beautifully appropriate tagline for Drasil
%       for a few reasons:
%                  1. What are ``Things''? One may only think of ``Things'' as far as their knowledge and understanding allows them! You wouldn't be able to think of things without some sort of basis/constructive understanding/methodology to _get_ there, you can't _think of random phenomena_ (that's why they're phenomena).
    
    \begin{itemize}
        \item<2-> An exploration in software-related artifact generation for ``well understood'' domains \cite{KnowledgeCapture2021} through strong knowledge capture.
            \begin{itemize}
                \item<3-> By unifying knowledge into a single framework with reusable composable units of knowledge, we eliminate code duplication, formally impose traceability and maintainability of knowledge (and software), and allow for easy knowledge transference.
                \item<4-> Knowledge organization and capture is of utmost importance, as it is the pathway for interpreters and Domain-Specific Languages (DSLs) to make appropriate usage of the knowledge captured.
                \item<5-> By creating different kinds of ``printers'', we can use a stable knowledge-base to generate software that solves ``well understood'' problems.
            \end{itemize}
        \item<6-> Drasil currently focuses on building research software, generating Software Requirement Specification documents (SRS) in both LaTeX and HTML (with MathJaX), code to solve a problem, README files, Makefiles, graphs, etc.
    \end{itemize}
\end{frame}

\begin{frame}
    \frametitle{Drasil Case Studies}
    \begin{itemize}
        \item<2-> Drasil currently contains a significant amount of Physics-related knowledge.
        \item<3-> As of writing, current case studies\footnotemark[1] are primarily related to physics, including:
            \begin{itemize}
                \item<4-> \textbf{GlassBR} - Predicting whether or not a glass slab is likely to resist a specified blast.
                \item<5-> \textbf{Single Pendulum} - Observing the motion of a single pendulum.
                \item<6-> \textbf{Double Pendulum} - Observing the motion of a double pendulum.
                \item<7-> \textbf{Game Physics} - Modelling of an open source 2D rigid body physics library used for games.
                \item<8-> \textbf{Proportional Derivative Controller (PDController)} - Examining the output of a ``Power Plant'' (Process Variable) over time.
                \item<9-> \textbf{Solar Water Heating System (SWHS)} - Modelling of a solar water heating system with phase change material, predicting temperatures and change in heat energy of water and the PCM over time.
            \end{itemize}
    \end{itemize}

    \footnotetext[1]{\tiny \url{https://jacquescarette.github.io/Drasil/\#Sec:Examples}}
\end{frame}

\begin{frame}
    \frametitle{Drasil Case Studies}
    \begin{itemize}
        \item<1-> \emph{cont.d}\footnotemark[1]:
            \begin{itemize}
                \item<1-> \textbf{SWHS without Phase Change Material (NoPCM)} - Modelling of a solar water heating system without phase change material, predicting temperatures and change in heat energy of water and the PCM over time.
                \item<2-> \textbf{Projectile} - Determining if a launched projectile hits a target, assuming no flight collisions.
                \item<3-> \textbf{Slope Stability Analysis Program (SSP)} - Assessment of the safety of a slope (composed of rock and soil) subject to gravity, identifying the surface most likely to experience slip and an index of its relative stability (factor of safety).
                \item<4-> \textbf{Heat Transfer Coefficients between Fuel and Cladding in Fuel Rods (HGHC)} - Examining the heat transfer coefficients related to clad.
            \end{itemize}
    \end{itemize}

    \onslide<5->{\textbf{The Drasil website is also generated by Drasil!}}

    \footnotetext[1]{\tiny \url{https://jacquescarette.github.io/Drasil/\#Sec:Examples}}
\end{frame}

\begin{frame}
    \frametitle{Taking a closer look at one of the examples: GlassBR}
    \framesubtitle{GlassBR Generates Code!}
    \begin{figure}
        \center
        \includegraphics[width=0.75\textwidth]{assets/DrasilSupportsChange.png}
        \caption{Knowledge flow from ``knowledge-base''/source to artifacts, by Dr. Spencer Smith}
        \label{fig:glassbr}
    \end{figure}
\end{frame}

\begin{frame}
    \frametitle{Which case studies currently generate code?}

    \begin{itemize}
        \item<2-> \textbf{GlassBR} - Predicting whether or not a glass slab is likely to resist a specified blast.
        \item<3-> \textbf{Proportional Derivative Controller (PDController)} - Examining the output of a ``Power Plant'' (Process Variable) over time.
        \item<4-> \textbf{SWHS without Phase Change Material (NoPCM)} - Modelling of a solar water heating system without phase change material, predicting temperatures and change in heat energy of water and the PCM over time.
        \item<5-> \textbf{Projectile} - Determining if a launched projectile hits a target, assuming no flight collisions.
    \end{itemize}
\end{frame}

\begin{frame}
    \frametitle{Why don't all case studies generate software artifacts?}
    \framesubtitle{Where will I be contributing?}

    \onslide<2->{After all,}
    \begin{itemize}
        \item<3-> They're all covered under ``well understood'' domains!
        \item<4-> The SRS documents are generated!\\
    \end{itemize}

    \onslide<5->{Generating view-only data (e.g., SRS documents) is considerably easier than generating artifacts that are ``evaluated'' in some way or another (e.g., compilation/interpretation/static analysis). These are largely different ``printers''.}

    \onslide<6->{A few, notable, blocking problems:}
    \begin{itemize}
        \item<7-> Confidently generating usable software artifacts without strong type information places significant stress on developers, resulting in a higher likelihood of bugs in artifacts.
        \item<8-> Existing ``theories''/``*Models''\footnotemark[1] don't expose enough information. They must be enriched, so that we can better interact with, and understand them.
    \end{itemize}

    \footnotetext[1]{\tiny Terminology is currently being changed, but is not reflected in many documents yet.}
\end{frame}

\subsection{Goal \#1: Typed Expression Language}

\begin{frame}
    \frametitle{Goal \#1: Typed Expression Language}
    \framesubtitle{Problem Description}
    
    \begin{itemize}
        \item<2-> Ensure only admissible expressions are used in GOOL-supported languages, and that all expressions are coherent.
        \item<3-> We want to ease developer cognitive load when writing expressions, as they will need to ensure their expressions are coherent, or else various problems (type, syntax, etc) can occur at runtime (of generated software artifacts).
    \end{itemize}
\end{frame}

\begin{frame}
    \frametitle{Goal \#1: Typed Expression Language}
    \framesubtitle{What makes up a ``good'' solution?}
    
    \begin{itemize}
        \item<2-> Catches, within reason, all possible scenarios where an expression goes awry.
        \item<3-> Allows GOOL code generator to also become typed!
        \item<4-> Add extra functionality to existing expression languages safely, allowing for new data types to be introduced.
        \item<5-> Adding type information to expressions shouldn't be a burden!
        \item<6-> Decomposing/splitting vocabularies so that we can impose restrictions on allowed terms, while not causing problems for interoperability.
    \end{itemize}
\end{frame}

\begin{frame}
    \frametitle{Goal \#1: Typed Expression Language}
    \framesubtitle{Current Progression}
    
    \begin{itemize}
        \item<2-> Split core mathematical expression language (Expr) into 3 variants (Expr, ModelExpr, and CodeExpr)
            \begin{itemize}
                \item<3-> Expr is intended to be ``directly computable'' language (e.g., an advanced calculator), expected to have a \emph{total} conversion into code.
                \item<4-> Created ModelExpr, which contains all other terms we might want to express, but won't necessarily be directly convertible into code. There are still a few operations left in Expr that need to be moved over, however.
                    \begin{itemize}
                        \item<5-> Theories that rely on discussion of terms only found in ModelExpr may only have representational code generated if we have rich surrounding data (preferably also that generates said ``ModelExpr''s, see goal \#2).
                    \end{itemize}
                \item<6-> CodeExpr is a clone of Expr, with a few extra functionalities for GOOL.
                \item<7-> Created a ``typed tagless final'' \cite{Carette2009finally} smart constructor encoding for writing expressions in Expr, and/or ModelExpr.
            \end{itemize}
    \end{itemize}
\end{frame}

\begin{frame}
    \frametitle{Goal \#1: Typed Expression Language}
    \framesubtitle{What are the next steps?}
    
    \begin{itemize}
        \item<2-> Continue moving inadmissible terms from Expr into ModelExpr.
        \item<3-> Moving literals from Expr \& ModelExpr into their own small language, so that areas that want \emph{strictly} literals can also have stronger restrictions.
        \item<4-> Adjusting containers to allow for expressions with a type variable.
        \item<5-> Adding the final type signatures, using Haskell GADT syntax.
    \end{itemize}
\end{frame}

\subsection{Goal \#2: Theory Discrimination -- ``ModelKinds''}

\begin{frame}
    \frametitle{Goal \#2: Theory Discrimination -- ``ModelKinds''}
    \framesubtitle{Problem Description}
    
    \begin{itemize}
        \item<2-> ``RelationConcepts'' were heavily used in both displaying expressions, and code generation. They are essentially ``Relation''s (``Expr''s) with a natural language description of them.
        \item<3-> ``RelationConcept''s don't contain enough information on their own to be a core component usable in general code generation.
        \item<4-> If the ``shape'' of the expressions are not uniform, then writing more ``interpreters''/``views''/code generators for them required difficult pattern analysis. It's also not a total-conversion.
    \end{itemize}
\end{frame}

\begin{frame}
    \frametitle{Goal \#2: Theory Discrimination -- ``ModelKinds''}
    \framesubtitle{What makes up a ``good'' solution?}
    
    \begin{itemize}
        \item<2-> A good solution involves making the ``Relation''s a ``view'' of a more data-rich specialized container for each kind of ``Theory''/``*Model''.
        \item<3-> ``ModelKinds''
        \item<4-> By constructing our final data views through ``more steps'' (e.g., with more depth), we obtain a better understanding of our ``theories''/``*Model''s/``ModelKinds'', allowing us to do more with them.
        \item<5-> We should be able to easily add extra ``ModelKind'' variants.
    \end{itemize}
\end{frame}

\begin{frame}
    \frametitle{Goal \#2: Theory Discrimination -- ``ModelKinds''}
    \framesubtitle{Current Progression}
    
    \begin{itemize}
        \item<2-> All ``RelationConcepts'' have been replaced, with one of:
            \begin{itemize}
                \item<3-> EquationalModels: ``QDefinition''s
                \item<4-> EquationalRealms: ``MultiDefn''s
                \item<5-> EquationalConstraints: ``ConstraintSet''s
                \item<6-> DEModels: ``RelationConcept''s
                \item<7-> OthModels: ``RelationConcept''s
            \end{itemize}
        \item<8-> Considerable number of ``theories''/``*Models'' have been restructured, but there are still many that are pending classification.
            \begin{itemize}
                \item<9-> Most are best to be done once we have a typed expression language (so that we can better handle expressions that involve collections), and the rest are differential equation-related models (primarily Dong's domain).
            \end{itemize}
    \end{itemize}
\end{frame}

\begin{frame}
    \frametitle{Goal \#2: Theory Discrimination -- ``ModelKinds''}
    \framesubtitle{What are the next steps?}
    
    \begin{itemize}
        \item<2-> Understanding what kinds of needs we have for ``collections'', pushing this information back into the typed expression language (once that is fully typed), and then creating model containers for these models.
        \item<3-> For the differential equation-related models, we will need to build appropriate models for each possible kind.
    \end{itemize}
\end{frame}

%%%%%%%%%%%%%%%%%%%%%%%%%%%%%%%%%%%%%%%%%%%%%%%%%%%%%%%%%%%%%%%%%%%%%%%%%%%%%%%
%% ACKNOWLEDGEMENTS
%%%%%%%%%%%%%%%%%%%%%%%%%%%%%%%%%%%%%%%%%%%%%%%%%%%%%%%%%%%%%%%%%%%%%%%%%%%%%%%

\begin{frame}
    \frametitle{Acknowledgements}

    \begin{itemize}
        \item<2-> Both goals are as designated by Dr. Carette, Dr. Smith, and past (and present) Drasil authors.
        \item<3-> ``ModelKinds'' is based on Dr. Carette's implementation.
        \item<4-> Dr. Smith created the GlassBR figure earlier shown.
        \item<5-> Drasil has had significant development by past (and present) authors. Their public notes in the issue tracker, and their works (in particular, that of Brooks' thesis) have been especially helpful in learning about Drasil.
    \end{itemize}
\end{frame}

%%%%%%%%%%%%%%%%%%%%%%%%%%%%%%%%%%%%%%%%%%%%%%%%%%%%%%%%%%%%%%%%%%%%%%%%%%%%%%%
%% A FINAL THANK YOU
%%%%%%%%%%%%%%%%%%%%%%%%%%%%%%%%%%%%%%%%%%%%%%%%%%%%%%%%%%%%%%%%%%%%%%%%%%%%%%%

\begin{frame}
    \center
    \huge{Fin.}\\
    \normalsize{Thank you!}
\end{frame}

%%%%%%%%%%%%%%%%%%%%%%%%%%%%%%%%%%%%%%%%%%%%%%%%%%%%%%%%%%%%%%%%%%%%%%%%%%%%%%%
%% REFERENCES
%%%%%%%%%%%%%%%%%%%%%%%%%%%%%%%%%%%%%%%%%%%%%%%%%%%%%%%%%%%%%%%%%%%%%%%%%%%%%%%

\section{References}

\begin{frame}[allowframebreaks]
    \frametitle{References}

    \bibliography{references}
    \bibliographystyle{apalike}
\end{frame}

\end{document}
