\chapter{Conclusion}
\label{chap:conclusion}

In this thesis, we addressed four (4) research questions, as stated in
\Cref{chap:intro:sec:problem-statement}:

\begin{enumerate}

    \item[\textbf{RQ1}] \textit{Drasil's current encoding of ``theories'' are
          essentially black boxes. We would like to use structural information
          present in the short list of the ``kinds'' of theories that show up in
          scientific computing. How do we codify that?}

\end{enumerate}

In \Cref{chap:modelkinds}, we replaced the ``black boxes'' with structured
versions of the same meaningful theories. In doing so, we opened up
opportunities for more domain-specific interpretation of same theories, such as
analysis, flexible printing, and most importantly, code generation. Chen's work
\cite{Chen2022MEng} has anecdotally shown the success of this work. Furthermore
in \Cref{chap:more-theory-kinds}, we began dissecting Drasil's currently encoded
theories, and replaced their encodings with structured variants in hopes of
further usage in code generation in the future.

\begin{enumerate}
    
    \item[\textbf{RQ2}] \textit{Drasil's theory encodings rely on a single
          mathematical expression language, and does not expose information
          about applicability to different contexts. In each context (e.g.,
          code, theories, and common arithmetic), certain terms of the
          expression language should be treated differently, or are simply
          inapplicable. How can we restrict term usage by context?}

\end{enumerate}

In \Cref{chap:lang-division}, we analyzed Drasil's single mathematical
expression language, divided it according to Drasil's current needs, and created
a means of using the divided variants seamlessly through creating a \acs{ttf}
encoding of its smart constructors. In doing this, we were able to restrict the
mathematical expressions admitted in concrete theories to only those with
definite values, which we can unambiguously convert into code fragments.

\begin{enumerate}

    \item[\textbf{RQ3}] \textit{How can we ensure that our mathematical
          expression language admits only valid expressions?}

\end{enumerate}

In \Cref{chap:typed-expr}, we began creating a system of type-rules that our
concrete mathematical expressions (\Expr{}) must obey. To enforce the type-rules
in Drasil, we built a bidirectional type-checker that runs on a whole \acs{srs}
and reports any errors it finds. We described the bidirectional type-checking
rules under a typeclass, so that we can later also add typing rules and
enforcement to the other expression languages.

\begin{enumerate}

    \item[\textbf{RQ4}] \textit{Our current ``typed'' approach to collecting
          different kinds of data is difficult to extend. How can we make it
          easier to extend?}

\end{enumerate}

In \Cref{chap:storingChunks}, we created a chunk database structure that is
capable of collecting \textit{any} chunk that conforms to a basic set of
requirements (e.g., has \acs{uid} and references to required chunks, by
\acs{uid}).

While there is a considerable amount of future work to be done
(\Cref{chap:future-work}), I am certain there will be an endless supply of
``future work'' after those tasks. Thus, with those leftover tasks in mind, I
pause for reflection.

The entirety of \Cref{chap:ideology} is one of the most important things I've
learned by conducting this research. Another important thing I've learned is
that \textit{actual} communication of knowledge from one to others is far more
complicated than it seems, and that any piece of knowledge has considerable
nuance to it. Finally, by attempting to codify, and act on, our understandings
of things, we're able to test just how ``well-understood''
\cite{well-understood} they truly are to us.
